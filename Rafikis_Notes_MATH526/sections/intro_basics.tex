Data, dara and deitar; in short, statistics is the art of learning from data. We usually start by \textbf{considering a question or problem} with our goal in mind being to \textbf{solve this by using
 data}. Our main tasks are to provide enough evidence in order to address the question or problem at hand and ideally we'd love to \textbf{draw a conclusion} or \textbf{reach a decision}.

 \vspace{0.2cm}

 In this course, we will merely discuss \textbf{how express the evidence} attained from the data given (usually, in most cases, we recieve really shitty data), but we depend on the quality, choice and methodology of
 the data collection, which really impacts our evidence and clarity of decision.

 \begin{remark}
    The collection of data usually has \textbf{structure} (i.g. patterns) and \textbf{`random' variation} (i.g. noise), hence, any conclusion we might reach is \textit{unbcertain}; therefore called an \textit{inference}.
 \end{remark}

 \vspace{0.2cm}

 So, a \textit{statistical inference} aims to use probability theory (big boy language, mathematical framework) in order to explain the variation in the data and to quantify the uncertainty in our conclusions.